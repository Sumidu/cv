%%%%%%%%%%%%%%%%%%%%%%%%%%%%%%%%%%%%%%%%%
% "ModernCV" CV and Cover Letter
% LaTeX Template
% Version 1.3 (29/10/16)
%
% This template has been downloaded from:
% http://www.LaTeXTemplates.com
%
% Original author:
% Xavier Danaux (xdanaux@gmail.com) with modifications by:
% Vel (vel@latextemplates.com)
%
% Adapted to be an Rmarkdown template by Mitchell O'Hara-Wild
% 23 November 2018
%
% License:
% CC BY-NC-SA 3.0 (http://creativecommons.org/licenses/by-nc-sa/3.0/)
%
% Important note:
% This template requires the moderncv.cls and .sty files to be in the same
% directory as this .tex file. These files provide the resume style and themes
% used for structuring the document.
%
%%%%%%%%%%%%%%%%%%%%%%%%%%%%%%%%%%%%%%%%%

%----------------------------------------------------------------------------------------
%	PACKAGES AND OTHER DOCUMENT CONFIGURATIONS
%----------------------------------------------------------------------------------------

\documentclass[11pt,a4paper,sans]{moderncv}

\usepackage[scaled=0.86]{DejaVuSansMono}

% Tweaks
\usepackage[german,ngerman]{babel}
\usepackage[T1]{fontenc}
\usepackage[utf8]{inputenc}
\usepackage{eurosym}

\moderncvstyle{casual} % CV theme - options include: 'casual' (default), 'classic', 'oldstyle' and 'banking'
\definecolor{color0}{rgb}{0,0,0}% black
\definecolor{color1}{HTML}{3873B3}% light blue
\definecolor{color2}{rgb}{0.45,0.45,0.45}% dark grey

\providecommand{\tightlist}{%
	\setlength{\itemsep}{0pt}\setlength{\parskip}{0pt}}

\usepackage[scale=0.75]{geometry} % Reduce document margins
\setlength{\hintscolumnwidth}{3cm} % Uncomment to change the width of the dates column
%\setlength{\makecvtitlenamewidth}{10cm} % For the 'classic' style, uncomment to adjust the width of the space allocated to your name



%----------------------------------------------------------------------------------------
%	NAME AND CONTACT INFORMATION SECTION
%----------------------------------------------------------------------------------------

\firstname{Dr.} % Your first name
\familyname{André Calero Valdez} % Your first name
\title{Junior Research Group Leader, AkadOR} % Job title/career
% \cvdate{November 2019} % Date of birth or date of CV??
\address{Maria-Theresia-Alle 19 -- 52064 Aachen, Germany}{} % Short address/location, use \newline if more than 1 line is required
 % Phone number
\mobile{+49 173-5963359} % Phone number
\homepage{http://calerovaldez.com}{calerovaldez.com} % Personal website
\email{\href{mailto:andrecalerovaldez@gmail.com}{\nolinkurl{andrecalerovaldez@gmail.com}}} % Email address

% \cvtwitter{\href{https://twitter.com/sumidu}{sumidu}}
% \cvgithub{\href{https://github.com/sumidu}{sumidu}}
% \cvlinkedin{}
\photo[70pt][0.4pt]{../data/photo.jpeg}
\extrainfo{
	\hspace*{1em}\href{https://twitter.com/sumidu}{twitter: sumidu}\hspace*{1em}	\hspace*{1em}\href{https://github.com/sumidu}{github: sumidu}\hspace*{1em}	
}


\quote{Human-Computer Interaction, Information Visualization, Recommender
Systems, Industry 4.0, eHealth, Social Implications, Technology
Acceptance}


% Templates for detailed entries
% Arguments: what when with where why
\usepackage{etoolbox}
\def\detaileditem#1#2#3#4#5{
	\cventry{#2}{#1}{#3}{#4}{}{{\ifx#5\empty\else{\begin{itemize}#5\end{itemize}}\fi}}}
\def\detailedsection#1{\nopagebreak#1}

% Templates for brief entries
% Arguments: what when with
\def\briefitem#1#2#3{\cvitem{#2}{#1 #3}}
\def\briefsection#1{\nopagebreak#1}

%----------------------------------------------------------------------------------------

\begin{document}

\makecvtitle % Print the CV title



\hypertarget{personal-information}{%
\section{Personal Information}\label{personal-information}}

\briefsection{\briefitem{22nd January 1981}{Date of Birth}{ }\briefitem{Married to Eva-Maria Peters (Teacher Sek2)}{Marital Status}{ }\briefitem{1 Daughter (2 years old)}{Children}{ }\briefitem{Maria-Theresia-Allee 19}{Address}{ }\briefitem{52064 Aachen, Germany}{}{ }\briefitem{andrecalerovaldez@gmail.com}{Email}{ }\briefitem{www.calerovaldez.com}{www}{ }}

\vspace{1em}

\hypertarget{scholarly-metrics}{%
\subsection{Scholarly Metrics}\label{scholarly-metrics}}

\briefsection{\briefitem{943}{Citations}{}\briefitem{17}{h-index}{}\briefitem{29}{i10-index}{}\briefitem{approx. \euro  1.3 Mio}{Acquired 3rd party Funding}{}}

\hypertarget{work-experience}{%
\section{Work Experience}\label{work-experience}}

\detailedsection{\detaileditem{Head of Consortium SmartIdentifikation}{2018--Present}{}{RWTH Aachen University, Germany}{\item{Collaborative Project: Germany/Austria}\item{Role: Project Lead. Team: 1 PhD students, 2 student assistants}}\detaileditem{Junior Research Group Leader}{2018--Present}{}{RWTH Aachen University, Germany}{\item{Junior research group Digitale Mündigkeit}\item{Role: Project Lead. Team: 3 PhD students, 2 student assistants}}\detaileditem{Guest Researcher}{2016}{Prof. Katrien Verbert}{KU Leuven, Belgium}{\item{HCI in Recommendation Systems}}\detaileditem{Post-Doc/Senior Researcher}{2016--Present}{Prof. Martina Ziefle}{Chair for Communication Science, RWTH Aachen University}{\item{Permanent contract. Akademischer Oberrat.}}\detaileditem{Post-Doc/Senior Researcher}{2014--2016}{Prof. Martina Ziefle}{Human-Computer Interaction Center, RWTH Aachen University}{\item{DFG-Project: EXC-128 - Integrative Production Technology for High Wage Countries}}\detaileditem{Visiting Professor}{2014--Present}{Prof. Andreas Holzinger}{Institut for medical informatics, statistics and documentation at the Medical University Graz, Austria}{\item{HCI in Knowledge Discovery and Machine Learning}}\detaileditem{Research Assistant}{2011--2014}{Prof. Martina Ziefle}{Human-Computer Interaction Center, RWTH Aachen University}{\item{DFG-Project: EXC-128}\item{BMBF-Project: iNec 01HH11045 - Innovation by Way of Experts Communities during Demographic Change}}\detaileditem{Research Assistant}{2009--2011}{Prof. Martina Ziefle}{Human-Technology Center (HumTec), RWTH Aachen University}{\item{BMBF-Project: Cairo}}\detaileditem{Computer Scientist}{2007--2008}{Atos Worldline}{Aachen, Germany}{\item{Establish code review guidelines for OWASP security}\item{Explore platform options for payment-platform (Java EE)}}}

\hypertarget{scientific-degrees}{%
\section{Scientific Degrees}\label{scientific-degrees}}

\detailedsection{\detaileditem{PhD Psychology}{2007--2008}{Faculty of Philosophy}{RWTH Aachen University}{\item{Topic:Technology Acceptance and Diabetes - User-centered design of small-screen devices for patients with diabetes}\item{Grade: summa cum laude (with highest honor)}}}

\hypertarget{education}{%
\section{Education}\label{education}}

\detailedsection{\detaileditem{Course of Study: Computer Science Diplom}{2001--2007}{Faculty of Mathematics, Computer Science, and Natural Sciences}{RWTH Aachen University, Germany}{\item{Major: Software-Construction}\item{Minor: Psychology}\item{Topic of Thesis: Design of a Benchmark and a tool for measuring performance of database access layers}\item{Grade: very good}}}

\hypertarget{teaching-experience}{%
\section{Teaching Experience}\label{teaching-experience}}

\detailedsection{\detaileditem{Doctoral Colloquium}{2018--Present}{every semester}{PhD Computer Science/Technical-Communication}{\item{In this course PhD students learn advanced research methods (systematic literature review, structural equation modeling, machine learning, agent-based modeling in Julia).}}\detaileditem{Agent-Based Modelling (seminar)}{2017}{once}{M.A. Digital Media Communication}{\item{In this seminar students learnt to programm social simulations in Netlogo and should address complex systems research questions in the context of social media.}}\detaileditem{Methods of technical communication (research plenary)}{2016--Present}{every semester}{B.Sc. Technikkommunikation}{\item{In this course students conduct reasearch in small groups and write short scientific articles. The topic changes every semester.}}\detaileditem{Visual Recommender Systems (proseminar)}{2016--Present}{every summer}{B.Sc. Computer Science}{\item{In this course students present the content of a scientific paper and write a short thesis about visual recommender systems.}}\detaileditem{Information Visualization and Language (seminar)}{2016--Present}{every summer}{M.A. Digital Media Communication}{\item{In this seminar students learn to programm interactive information visualizations using open data sets or text-mining tools.}\item{The programming language used in first iterations was Processing.}\item{Currently, the course uses R, RMarkdown, Shiny and GGPlot as a programming environment.}}\detaileditem{Advanced Empirical Methods for social scientists (lecture)}{2015--Present}{every winter}{M.A. Digital Media Communication}{\item{In this lecture students learn advanced methods of social empirical research using the R Language (systematic literature review, formulating a research question, measurement theory, data collection, statistical evaluation, reporting).}\item{Teaching assistant in 2015 and 2016. Held the lecture in 2017, 2018, and 2019.}\item{Transitioned the course from using SPSS to R for statistical analyses in 2018.}\item{This lecture is accompanied by a seminar, where students apply the theoretical knowledge on a real life scenario.}}\detaileditem{Application seminar (seminar)}{2015--Present}{every winter}{M.A. Digital Media Communication}{\item{Accompanying seminar to Advanced Empirical Methods for social scientists.}}\detaileditem{Interdisciplinary Publishing (graduate seminar)}{2014--Present}{every semester}{All PhD students at RWTH}{\item{In this seminar, aimed at doctoral students, participants learn about scientific publishing, scientific metrics, ethics in science and open access publishing.}}\detaileditem{Forms of representation in science (ring lecture)}{2013--Present}{every winter}{B.A. Linguistics and Communication Science}{\item{In this lecture students learn about the visual display of quantitative information and networked data.}}\detaileditem{Usability, User Diversity, Technology acceptance (seminar)}{2013--Present}{every winter}{M.A. Digital Media Communication, M.Sc. Technical Communication, M.Sc. Mobility and Logistics}{\item{In this seminar students try to answer a self-chosen research question.}\item{Students in this course may pick their methods (qualitative, quantitative).}\item{In 2018 half of the course chose social-media-mining methods to understand public discourse in media.}}\detaileditem{Bachelor colloquium (colloquium)}{2012--Present}{every winter + some summers when needed}{B.Sc. Technical communication}{\item{In this colloquium students learn the craft of scientific writing and how to structure a research report.}}\detaileditem{Interaction and Communication between Humans and Technology (seminar)}{2011--Present}{once}{MA Linguistics and Communication Science}{\item{In this seminar students jointly conducted research about social media and usage motivation.}}\detaileditem{Research internship (internship)}{2011--Present}{held twice}{B.Sc. Technical communication}{\item{In this full-time practice students had to develop their own small research question and collaborate with other participants to improve the effectiveness during a 4-week internship.}}\detaileditem{Webtechnologies (colloquium)}{2009--2010}{once}{M.Sc. Computer Science}{\item{In this colloquium students learnt to programm interactive websites using PHP and Java EE}}}

\vspace{1em}

\hypertarget{invited-guest-lectures}{%
\subsection{Invited Guest Lectures}\label{invited-guest-lectures}}

\detailedsection{\detaileditem{Quantiative Evaluation of Information Visualization}{2019-04-02}{BISS Brightlands Smart Services Campus - Zuyd Hogeschool}{Heerlen, NL}{\empty}\detaileditem{User-Centered Design of Information Visualization}{2019-02-26}{BISS Brightlands Smart Services Campus - Zuyd Hogeschool}{Heerlen, NL}{\empty}\detaileditem{Trump, Brexit und die AFD. Verändert Facebook unsere Demokratie?}{2018-11-29}{Rotary Aachen}{Aachen, Germany}{\empty}\detaileditem{Trump, Brexit und die AFD. Verändert Facebook unsere Demokratie?}{2018-11-17}{VVdst}{Aachen, Germany}{\empty}\detaileditem{Connecting Data with Humans}{2016-05-24}{}{KU Leuven, Belgium}{\empty}\detaileditem{Empirische Kommunikationswissenschaften}{2016-05-20}{Bundesfachschaftstagung der Kommunikationswissenschaften}{RWTH Aachen University, Germany}{\empty}}

\hypertarget{grants-awards-and-honors}{%
\section{Grants, Awards and Honors}\label{grants-awards-and-honors}}

\briefsection{\briefitem{Best Paper Award: “Hey, Siri”, “Ok, Google”, “Alexa”. Acceptance-relevant factors of virtual voice-assistants}{2019}{IEEE Conference on Professional Communication}\briefitem{Grant: E-Learning Project: ShinyStats, Funding volume 5,000.- Euro.}{2018}{RWTH Digitization Fund}\briefitem{Grant: SmartIdentifikation, Total funding volume 950,000.- Euro, Own share: 302,767.- Euro.}{2018}{BMBF}\briefitem{Best Poster Award: Political Opinions of Us and Them and the Influence of Digital Media Usage.}{2018}{EuroCSS 2013 Conference}\briefitem{Grant: Junior Research Group: Digitale Mündigkeit, Funding volume 949,500.- Euro.}{2017}{NRW}\briefitem{Grant: Getting around in Cyberspace. Together with Prof. Martina Ziefle, Prof. Leif Kobbelt, Funding volume (own share) 65,000.- Euro.}{2017}{RWTH-Aachen Profile Area for Information and Communication Technology}\briefitem{Award: Borchers-Insignia of the proRWTH e.v. for an outstandig doctoral dissertation RWTH Aachen University Doctoral Scholarship.}{2014}{RWTH Aachen University}\briefitem{Best Paper Award: The Impact of User Diversity on the Willingness to Disclose Personal Information in Social Network Services.}{2013}{SoutCHI 2013 Conference}\briefitem{Best Presentation Award: Personality Influences on Etiquette Requirements for Social Media in the Work Context.}{2013}{SoutCHI 2013 Conference}\briefitem{Best Paper Award: Openness to Accept Medical Technology - A Cultural View.}{2011}{USAB 2011 Conferences}\briefitem{Grant: RWTH Aachen University Doctoral Scholarship for Students with excellent degrees.}{2009}{RWTH Aachen University}}

\hypertarget{presentations}{%
\section{Presentations}\label{presentations}}

\hypertarget{talks}{%
\subsection{Talks}\label{talks}}

\detailedsection{\detaileditem{Towards an Understanding of Opinion Formation on the Internet - Using a Latent Process Model to Understand the Spread of Information on Social Media}{Sep 2019}{European Social Simulation Conference}{Mainz, DE}{\empty}\detaileditem{Trends and Changes in the Field of HCI the Last Decade from the Perspective of HCII Conference}{Jul 2019}{Human-Computer Interaction International}{Orlando, USA}{\empty}\detaileditem{Computational Methods in Professional Communication}{Jul 2019}{IEEE Professional Communication}{Aachen, Germany}{\empty}\detaileditem{Human and Algorithmic Contributions to Misinformation Online - Identifying the Culprit (Keynote)}{Feb 2019}{Misdoom Conference}{Hamburg}{\empty}\detaileditem{Industry 4.0 and its Future Staff. Matching Millennials Perceptions of a Perfect Job with the Requirements of Digitalization}{Jan 2019}{Competitive Manufacturing Conference (COMA)}{Stellenbosch, ZA}{\empty}\detaileditem{Predicting Acceptance of Novel Technology from Social Network Data-An Agent-based Simulation-Approach}{Jan 2019}{Competitive Manufacturing Conference (COMA)}{Stellenbosch, ZA}{\empty}\detaileditem{Requirements for Reproducibility of Geo-Spatial Visualizations (invited panel talk)}{Okt 2018}{IEEE VIS, Beliv Workshop}{Berlin, Germany}{\empty}\detaileditem{Human-Factors in the Age of Algorithms}{Jul 2018}{Human-Computer Interaction International}{Las Vegas, USA}{\empty}\detaileditem{Believability of Fake News}{Jul 2018}{Triennial Congress of the international ergonomics association (IEA)}{Florence, Italy}{\empty}\detaileditem{Acceptance of a paper-based digital diabetes diary}{Mär 2018}{ICT4AWE Conference}{Madeira, Portugal}{\empty}\detaileditem{E-Coaching for the elderly – Recommender Systems in Health (invited panel talk)}{Mär 2018}{ICT4AWE Conference}{Madeira, Portugal}{\empty}\detaileditem{Measuring Insight Into Multi-Dimensional Data from a Combination of a Scatterplot Matrix and a Hyperslice Visualization}{Jul 2017}{Human-Computer Interaction International 2017}{Vancouver, Canada}{\empty}\detaileditem{Industrie 4.0 and an Aging Workforce – A Discussion from a Psychological and a Managerial Perspective}{Jul 2017}{Human-Computer Interaction International 2017}{Vancouver, Canada}{\empty}\detaileditem{HCI for Recommender Systems, the Past, the Present and the Future}{Okt 2016}{ACM RecSys 2016}{MIT, Boston, USA}{\empty}\detaileditem{Human Factors in Information Visualization and Decision Support Systems}{Sep 2016}{Mensch und Computer 2016}{RWTH Aachen University, Germany}{\empty}\detaileditem{Visualizing Opportunities of Collaboration in Large Research Organizations}{Jul 2016}{Human-Computer Interaction International 2016}{Toronto, Kanada}{\empty}\detaileditem{Using Liferay as an Interdisiplinary Scientific Collaboration Portal}{Jul 2016}{Human-Computer Interaction International 2016}{Toronto, Kanada}{\empty}\detaileditem{Strategic Knowledge Management for Interdisciplinary Teams. Overcoming Barriers of Interdisciplinary Work via an Online Portal Approach}{Jul 2016}{Human-Computer Interaction International 2016}{Toronto, Kanada}{\empty}\detaileditem{Preparing Production Systems for the Internet of Things}{Jan 2016}{International Conference on Competitive Manufacturing (COMA)}{Stellenbosch, South Africa}{\empty}\detaileditem{Reducing Complexity with Simplicity}{Aug 2015}{International Ergonomics Association Triennial Summit}{Melbourne, Australia}{\empty}\detaileditem{What should I read next?}{Jul 2015}{Human-Computer Interaction International Conference}{Los Angeles, USA}{\empty}\detaileditem{What do my colleagues know?}{Jul 2015}{Human-Computer Interaction International Conference}{Los Angeles, USA}{\empty}\detaileditem{Older users rejection of mobile health apps}{Jul 2015}{Human-Computer Interaction International Conference}{Los Angeles, USA}{\empty}\detaileditem{On the visual design of enterprise resource systems}{Jul 2015}{Applied Human Factors and Ergonomics 2015}{Las Vegas, USA}{\empty}\detaileditem{Orchestrating collaboration - Visual collaboration suggestions in large research clusters}{Jul 2015}{Applied Human Factors and Ergonomics 2015}{Las Vegas, USA}{\empty}\detaileditem{User-centered design of business communities. The influence of user diversity on motivation to use}{Jul 2015}{Applied Human Factors and Ergonomics 2015}{Las Vegas, USA}{\empty}\detaileditem{Reasons for using social networks professionally}{Jul 2014}{Human Computer Interaction International 2014}{Crete, Greece}{\empty}\detaileditem{Enhancing interdisciplinary collaboration in large research clusters}{Jul 2014}{Human Computer Interaction International 2014}{Crete, Greece}{\empty}\detaileditem{Scientific Cooperation Engineering}{Okt 2013}{10th International Conference on Webometrics, Informetrics, and Scientometrics}{Illmenau, Germany}{\empty}\detaileditem{Graph entropy measures in publication network data}{Sep 2013}{CD-ARES 2013}{Regensburg, Germany}{\empty}\detaileditem{Information disclosure in social networks}{Jun 2013}{SouthChi-Conference 2013}{Maribor, Slovenia}{\item{Best presentation award}}\detaileditem{Netiquette for the work environment}{Jun 2013}{SouthChi-Conference 2013}{Maribor, Slovenia}{\empty}\detaileditem{Social Networking Sites in the work environment}{Feb 2012}{International Ergonomics Association Triennial Summit}{Recife, Brazil}{\empty}\detaileditem{Mental Models of Menu Structures in Diabetes Assistants}{Jul 2010}{12th ICCHP-International Conference on Computers helping people with special needs}{Vienna, Austria}{\empty}\detaileditem{Task-Performance in mobile and ambient interfaces for an electronic diabetes assistant}{Jun 2010}{iSociety 2010}{London, United Kingdom}{\empty}\detaileditem{Effects of Aging and Domain Knowledge on Usability in a Diabetes Small Screen Device}{Jul 2009}{USAB 2009}{Linz, Austria}{\empty}}

\hypertarget{service}{%
\section{Service}\label{service}}

\hypertarget{memberships}{%
\subsection{Memberships}\label{memberships}}

\briefsection{\briefitem{Member of Gesellschaft für Informatik (GI)}{}{}\briefitem{Member of Association for Computing Machinery (ACM)}{}{}\briefitem{Member of Institute of Electrical and Electronics Engineers (IEEE)}{}{}\briefitem{Member of Technical Comittee: Work with Computer Systems (WWCS) of the International Ergonomics Association (IEA)}{}{}\briefitem{Member Technical Comittee: Visual Ergonomics of the International Ergonomics Association (IEA)}{}{}\briefitem{Member of the Expert-Network: Human-Computer Interaction in Knowledge Discovery and Data-Mining (HCI-KDD)}{}{}\briefitem{Member of the European Social Simulation Association (ESSA)}{}{}}

\hypertarget{manuscript-reviews}{%
\subsection{Manuscript Reviews}\label{manuscript-reviews}}

\briefsection{\briefitem{Computers in Human Behavior, Interacting With Computers, International Journal of Human-Computer Studies, ACM RecSys, ACM SIGCHI, ACM MobileHCI, ACM TVX, IEEE VIS, IEEE Transactions on Visualization and Computer Graphics, IEEE Transactions on Knowledge and Data Engineering, WirtschaftsInformatik, BMC Medical Informatics and Decision Making, PlosOne, Political Psychology, International Journal of Health Policy, MDPI Journal Mathematics, MDPI Journal Entropy, MDPI Informatics, Journal Online Information Review, Journal Convergence, Pervasive Health Conference, Mensch und Computer Conference, Human-Computer Interaction International Conference, USAB Conference, South-Chi Conference}{Adhoc reviews}{}\briefitem{ACM SigCHI Associate Editor Late Breaking Results, Digital Human Modeling Conference, BELIV Workshops at IEEE VIS, WirtschaftsInformatik(WI), HealthRecSys Workshops at ACM RecSys, UMAP Conference, EnChiRes Workshops at ACM EICS, CD-MAKE Conference at ARES, USAB Conference, HCI-KDD Workshops, HFIDSS-Workshops}{Editorial-Board or Program Committee}{}\briefitem{Human Factors in Information Visualization (HFIDSS), Workshops at the Mensch und Computer and HCI International Conferences}{Head of Program Committee}{}}

\hypertarget{academic-services}{%
\subsection{Academic Services}\label{academic-services}}

\detailedsection{\detaileditem{Faculty Tenure-Track Commission}{2019--Present}{RWTH Aachen University}{Elected Member}{\item{Determine tenure track evaluation criteria}}\detaileditem{Teaching-Coordinator}{2017--Present}{Course of Study: Digital Media Communication (M.A.)}{Institute for Linguistics and Communication Science, RWTH Aachen University}{\item{Responsible for managing online catalogues, tracking of teaching duties, Exam management, Student troubleshooting}}\detaileditem{Coordinator International Office}{2016--Present}{Institute for Linguistics and Communication Science}{RWTH Aachen University}{\item{Responsible for Erasmus+ exchange students and staff mobility}\item{Coordination of all other international student activities}}\detaileditem{Committee work}{2016}{Professorship Computational methods in Social Sciences (W3)}{Faculty of Philosophy, RWTH Aachen University}{\item{Elected committee member}}\detaileditem{Committee work}{2016}{Professorship Philosophy of Science (W3)}{Faculty of Philosophy, RWTH Aachen University}{\item{Elected deputy committee member}}\detaileditem{Committee work}{2017}{Professorship Auditive Cognition (W2)}{Faculty of Philosophy, RWTH Aachen University}{\item{Elected committee member}}\detaileditem{Accreditation work}{2016}{Course of Study: Digital Media Communication (M.A.)}{Institute for Linguistics and communication science, RWTH Aachen University}{\item{Responsible for the preparation of the module manual and the study plan.}}\detaileditem{Grant Proposal Review}{2019}{Commission for scientific evaluation}{ETH Zürich, Switzerland}{\item{Review of International Large Scale Research Proposal}}\detaileditem{Advisory}{2018}{BMBF/VDI/VDE Innovation + Technik GmbH}{Stuttgart}{\item{Invited Talk: Future Topics in HCI Research}\item{Advisory for future BMBF programme concepts}}\detaileditem{Advisory}{2018}{BMBF/Deutsches Zentrum für Luft und Raumfahrt (DLR) e.V.}{Bonn}{\item{Advisory for 9th Framework EU programme concepts}}}

\hypertarget{bibliography}{%
\section{Bibliography}\label{bibliography}}

\textbf{Articles published by outlets with scientific quality assurance,
book publications, and works accepted for publication but not yet
published.}

Citations are retrieved from Google Scholar on 23th November 2019.

\begin{enumerate}
\def\labelenumi{\arabic{enumi}.}
\tightlist
\item
  Laura Burbach, Poornima Belavadi, Nils Plettenberg, Johannes Nakayama,
  \textbf{André Calero Valdez} (2020). Using agent-based modeling to
  simulate the influence of different recommender systems on news
  exposure and thematic polarization. In: \emph{Communication Methods
  and Measures} \textbf{(submitted)}
\item
  \textbf{André Calero Valdez}, M Ziefle (2019). The users' perspective
  on the privacy-utility trade-offs in health recommender systems. In:
  \emph{International Journal of Human-Computer Studies}, 121, 108-121
  (cited 21 times)
\item
  P Brauner, R Philipsen, \textbf{André Calero Valdez}, M Ziefle (2019).
  What happens when decision support systems fail?---the importance of
  usability on performance in erroneous systems. In: \emph{Behaviour \&
  Information Technology}, 1-18 (cited 2 times)
\item
  \textbf{André Calero Valdez}, M Ziefle (2019). Predicting Acceptance
  of Novel Technology from Social Network Data-An Agent-based
  Simulation-Approach. In: \emph{International Conference on Competitive
  Manufacturing} (cited 1 times)
\item
  P Halbach, L Burbach, M Ziefle, \textbf{André Calero Valdez} (2019).
  Follow Me: The Impact of Opinion Majorities in Social Networks and the
  Role of Digital Maturity. In: \emph{International Conference on
  Human-Computer Interaction}, 488-500
\item
  L Burbach, P Halbach, M Ziefle, \textbf{André Calero Valdez} (2019).
  Bubble Trouble: Strategies Against Filter Bubbles in Online Social
  Networks. In: \emph{International Conference on Human-Computer
  Interaction}, 441-456
\item
  \textbf{André Calero Valdez}, M Ziefle (2019). Trends and Changes in
  the Field of HCI the Last Decade from the Perspective of HCII
  Conference. In: \emph{International Conference on Human-Computer
  Interaction}, 31-45
\item
  L Burbach, P Halbach, N Plettenberg, J Nakayama, M Ziefle,
  \textbf{André Calero Valdez} (2019). " Hey, Siri``,'' Ok, Google``,''
  Alexa". Acceptance-Relevant Factors of Virtual Voice-Assistants. ,
  2019 IEEE International Professional Communication Conference
  (ProComm), 101-111
\item
  P Halbach, L Burbach, J Nakayama, N Plettenberg, M Ziefle,
  \textbf{André Calero Valdez} (2019). Would I Lie to You? How Users
  Evaluate Faked Online Content Depending on Its Publication Type. ,
  2019 IEEE International Professional Communication Conference
  (ProComm), 56-64
\item
  \textbf{André Calero Valdez}, L Adam, D Assenmacher, L Burbach, M
  Bonart, L Frischlich, et al. (2019). Computational Methods in
  Professional Communication. , 2019 IEEE International Professional
  Communication Conference (ProComm), 275-285
\item
  J Nakayama, N Plettenberg, P Halbach, L Burbach, M Ziefle,
  \textbf{André Calero Valdez} (2019). Trust in Cyber Security
  Recommendations. , 2019 IEEE International Professional Communication
  Conference (ProComm), 48-55
\item
  L Burbach, P Halbach, M Ziefle, \textbf{André Calero Valdez} (2019).
  Who Shares Fake News in Online Social Networks?. In: \emph{Proceedings
  of the}, 27th ACM Conference on User Modeling, Adaptation and~
\item
  LS Vervier, \textbf{André Calero Valdez}, M Ziefle (2019).
  ``Attitude''-mHealth Apps and Users' Insights: An Empirical Approach
  to Understand the Antecedents of Attitudes towards mHealth
  Applications. In: \emph{ICT}, T4AWE - 5th International Conference on
  Information and Communication~
\item
  AK Schaar, \textbf{André Calero Valdez}, T Hamann, M Ziefle (2019).
  INDUSTRY 4.0 AND ITS FUTURE STAFF. MATCHING MILLENNIALS PERCEPTIONS OF
  A PERFECT JOB WITH THE REQUIREMENTS OF DIGITALIZATION. In:
  \emph{International Conference on Competitive Manufacturing}
\item
  \textbf{André Calero Valdez} (2019). Human and Algorithmic
  Contributions to Misinformation Online - Identifying the Culprit. In:
  \emph{MISDOOM Conference} \textbf{(accepted)}
\item
  Laura Burbach, Patrick Halbach, Nils Plettenberg, Johannes Nakayama,
  Martina Ziefle, \textbf{André Calero Valdez} (2019). Towards an
  Understanding of Opinion Formation on the Internet - Using a Latent
  Process Model to Understand the Spread of Information on Social Media.
  In: \emph{European Conference on Social Simulation ESSA}
  \textbf{(accepted)}
\item
  P Schmitz, J Hildebrandt, \textbf{André Calero Valdez}, L Kobbelt, M
  Ziefle (2018). You spin my head right round: Threshold of limited
  immersion for rotation gains in redirected walking. In: \emph{IEEE
  transactions on visualization and computer graphics}, 24 (4),
  1623-1632 (cited 15 times)
\item
  L Burbach, J Nakayama, N Plettenberg, M Ziefle, \textbf{André Calero
  Valdez} (2018). User preferences in recommendation algorithms: the
  influence of user diversity, trust, and product category on privacy
  perceptions in recommender algorithms. In: \emph{Proceedings of the},
  12th ACM Conference on Recommender Systems, 306-310 (cited 5 times)
\item
  \textbf{André Calero Valdez}, J Brell, AK Schaar, M Ziefle (2018). The
  diversity of why: a meta-analytical study of usage motivation in
  enterprise social networks. In: \emph{Universal Access in the
  Information Society}, 17 (3), 549-566 (cited 5 times)
\item
  \textbf{André Calero Valdez}, M Ziefle (2018). Believability of News.
  In: \emph{Congress of the International Ergonomics Association},
  469-477 (cited 2 times)
\item
  \textbf{André Calero Valdez}, M Ziefle (2018). Human Factors in the
  Age of Algorithms. Understanding the Human-in-the-loop Using
  Agent-Based Modeling. In: \emph{International Conference on Social
  Computing and Social Media}, 357-371 (cited 2 times)
\item
  \textbf{André Calero Valdez}, L Burbach, M Ziefle (2018). Political
  Opinions of Us and Them and the Influence of Digital Media Usage. In:
  \emph{International Conference on Social Computing and Social Media},
  189-202 (cited 2 times)
\item
  M Ziefle, \textbf{André Calero Valdez} (2018). Decisions About Medical
  Data Disclosure in the Internet: An Age Perspective. In:
  \emph{International Conference on Human Aspects of IT for the Aged
  Population}, 186-201 (cited 2 times)
\item
  J Hildebrandt, P Schmitz, \textbf{André Calero Valdez}, L Kobbelt, M
  Ziefle (2018). Get Well Soon! Human Factors' Influence on
  Cybersickness after Redirected Walking Exposure in Virtual Reality.
  In: \emph{International Conference on Virtual, Augmented and Mixed
  Reality}, 82-101 (cited 2 times)
\item
  \textbf{André Calero Valdez}, J Kluge, M Ziefle (2018). Elitism,
  trust, opinion leadership and politics in social protests in Germany.
  In: \emph{Energy Research \& Social Science}, 43, 132-143 (cited 1
  times)
\item
  L Vervier, \textbf{André Calero Valdez}, M Ziefle (2018). ``Should I
  Trust or Should I Go?'' or What Makes Health-Related Websites Appear
  Trustworthy?-An Empirical Approach of Perceived Credibility of Digital
  Health Information and the Influence of User Diversity. In:
  \emph{International Conference on Information and Communication
  Technologies for Ageing Well and e-Health} (cited 1 times)
\item
  \textbf{André Calero Valdez}, AK Schaar, JR Hildebrandt, M Ziefle
  (2018). Requirements for Reproducibility of Research in Situational
  and Spatio-Temporal Visualization. In: \emph{BELIV Workshop at VIS},
  S2018 (cited 1 times)
\item
  J Nakayama, N Plettenberg, L Burbach, \textbf{André Calero Valdez}
  (2018). Similarity of Search Results in the Datenspende BTW17
  Dataset-CSS Data Challenge 2018. In: \emph{EuroCSS}, 2018 2
\item
  S Müller-Abdelrazeq, P Brauner, \textbf{André Calero Valdez}, U
  Jansen, L Platte, et al. (2018). Interdisciplinary Cooperation
  Management in Research Clusters: A Review of Twelve Years. In:
  \emph{ICICKM}, 2018 15th International Conference on Intellectual
  Capital Knowledge~
\item
  P Brauner, R Philipsen, \textbf{André Calero Valdez}, M Ziefle (2018).
  Human Interaction Under Risk in Cyber-Physical Production Systems. In:
  \emph{Congress of the International Ergonomics Association}, 421-430
\item
  R Philipsen, P Brauner, \textbf{André Calero Valdez}, M Ziefle (2018).
  Evaluating Strategies to Restore Trust in Decision Support Systems in
  Cross-Company Cooperation. In: \emph{International Conference on
  Applied Human Factors and Ergonomics}, 115-126
\item
  \textbf{André Calero Valdez}, M Ziefle (2018). Studying the Acceptance
  of a Digital Diabetes Diaries. In: \emph{International Conference on
  Information and Communication Technologies for Ageing Well and
  e-Health}
\item
  L Vervier, \textbf{André Calero Valdez}, M Ziefle (2018). ``In Clarity
  We Trust!''-An Empirical Study of Factors that Affect the Credibility
  of Health-Related Information on Websites. In: \emph{International
  Conference on Information and Communication Technologies for Ageing
  Well and e-Health}
\item
  \textbf{André Calero Valdez}, M Ziefle, M Sedlmair (2018). Studying
  Biases in Visualization Research: Framework and Methods. In:
  \emph{Cognitive Biases in Visualizations}, 13-27
\item
  \textbf{André Calero Valdez}, M Ziefle (2018). Acceptance of a Digital
  Paper-based Diabetes Diary. , 4th International Conference on
  Information and Communication Technologies~
\item
  H Schäfer, S Hors-Fraile, RP Karumur, \textbf{André Calero Valdez}, A
  Said, et al. (2017). Towards health (aware) recommender systems. In:
  \emph{Proceedings of the}, 2017 international conference on digital
  health, 157-161 (cited 34 times)
\item
  \textbf{André Calero Valdez}, M Ziefle, M Sedlmair (2017). Priming and
  anchoring effects in visualization. In: \emph{IEEE transactions on
  visualization and computer graphics}, 24 (1), 584-594 (cited 26 times)
\item
  \textbf{André Calero Valdez}, M Ziefle, M Sedlmair (2017). A Framework
  for Studying Biases in Visualization Research. In: \emph{DECISIVe},
  2017 2 (cited 12 times)
\item
  M Schinner, \textbf{André Calero Valdez}, E Noll, AK Schaar, P
  Letmathe, M Ziefle (2017). `Industrie 4.0'and an Aging Workforce--A
  Discussion from a Psychological and a Managerial Perspective. In:
  \emph{International Conference on Human Aspects of IT for the Aged
  Population}, 537-556 (cited 5 times)
\item
  P Brauner, \textbf{André Calero Valdez}, R Philipsen, M Ziefle (2017).
  How correct and defect decision support systems influence trust,
  compliance, and performance in supply chain and quality management.
  In: \emph{International Conference on HCI in Business, Government, and
  Organizations} (cited 5 times)
\item
  M Ziefle, \textbf{André Calero Valdez} (2017). Domestic robots for
  homecare: a technology acceptance perspective. In: \emph{International
  Conference on Human Aspects of IT for the Aged Population}, 57-74
  (cited 4 times)
\item
  D Elsweiler, S Hors-Fraile, B Ludwig, A Said, H Schäfer, C Trattner,
  et al. (2017). Second workshop on health recommender
  systems:(healthrecsys 2017). In: \emph{Proceedings of the Eleventh ACM
  Conference on Recommender Systems}, 374-375 (cited 2 times)
\item
  P Brauner, R Philipsen, \textbf{André Calero Valdez}, M Ziefle (2017).
  The Interaction of Causal Attribution of Performance and Compliance
  with Decision Support Systems in Cyber-Physical Production Systems-An
  Empirical Study Using a Business~\ldots{}. In: \emph{International
  Conference on Applied Human Factors and Ergonomics}, 11-23 (cited 2
  times)
\item
  C Mendez, V Lukarov, C Greven, \textbf{André Calero Valdez}, F Dietze,
  U Schroeder, et al. (2017). User Groups and Different Levels of
  Control in Recommender Systems. In: \emph{International Conference on
  Digital Human Modeling and Applications in Health, Safety, Ergonomics
  and Risk Management} (cited 2 times)
\item
  T Thiele, \textbf{André Calero Valdez}, S Stiehm, A Richert, M Ziefle,
  S Jeschke (2017). Augmenting research cooperation in production
  engineering with data analytics. In: \emph{Production Engineering}, 11
  (2), 213-220 (cited 2 times)
\item
  \textbf{André Calero Valdez}, S Gebhardt, TW Kuhlen, M Ziefle (2017).
  Measuring Insight into Multi-dimensional Data from a Combination of a
  Scatterplot Matrix and a HyperSlice Visualization. In:
  \emph{International Conference on Digital Human Modeling and
  Applications in Health, Safety, Ergonomics and Risk Management} (cited
  1 times)
\item
  M Bukowski, \textbf{André Calero Valdez}, M Ziefle, T Schmitz-Rode, R
  Farkas (2017). Hybrid collaboration recommendation from bibliometric
  data. In: \emph{Second International Workshop on Health Recommender
  Systems}, 36
\item
  F Dietze, \textbf{André Calero Valdez}, J Karoff, C Greven, U
  Schroeder, M Ziefle (2017). That's so Meta! Usability of a
  Hypergraph-Based Discussion Model. In: \emph{International Conference
  on Digital Human Modeling and Applications in Health, Safety,
  Ergonomics and Risk Management}
\item
  S Jeschke, W Bleck, A Richert, G Schuh, W Schulz, M Ziefle, A
  Bräkling, et al. (2017). Scientific Cooperation Engineering. In:
  \emph{Integrative Production Technology}, 993-1046
\item
  \textbf{André Calero Valdez}, M Ziefle, K Verbert, A Felfernig, A
  Holzinger (2016). Recommender systems for health informatics:
  state-of-the-art and future perspectives. In: \emph{Machine Learning
  for Health Informatics}, 391-414 (cited 47 times)
\item
  \textbf{André Calero Valdez}, M Ziefle, K Verbert (2016). HCI for
  recommender systems: the past, the present and the future. In:
  \emph{Proceedings of the}, 10th ACM Conference on Recommender Systems,
  123-126 (cited 24 times)
\item
  J Brell, \textbf{André Calero Valdez}, AK Schaar, M Ziefle (2016).
  Gender differences in usage motivation for social networks at work.
  In: \emph{International Conference on Learning and Collaboration
  Technologies}, 663-674 (cited 12 times)
\item
  T Vaegs, \textbf{André Calero Valdez}, AK Schaar, A Breakling, S
  Aghassi, U Jansen, et al. (2016). Enhancing scientific cooperation of
  an interdisciplinary cluster of excellence via a scientific
  cooperation portal. In: \emph{Automation, Communication and
  Cybernetics in Science and Engineering}, 2015~ (cited 11 times)
\item
  F Dietze, J Karoff, \textbf{André Calero Valdez}, M Ziefle, C Greven,
  U Schroeder (2016). An open-source object-graph-mapping framework for
  Neo4j and scala: renesca. In: \emph{IFIP WG}, 8.4, 8.9, TC 5
  International Cross-Domain Conference, CD-ARES 2016~ (cited 9 times)
\item
  P Brauner, \textbf{André Calero Valdez}, R Philipsen, M Ziefle (2016).
  Defective still deflective--how correctness of decision support
  systems influences user's performance in production environments. In:
  \emph{International Conference on HCI in Business, Government, and
  Organizations} (cited 8 times)
\item
  A Holzinger, \textbf{André Calero Valdez}, M Ziefle (2016). Towards
  interactive recommender systems with the doctor-in-the-loop. In:
  \emph{Mensch und Computer}, 2016--Workshopband (cited 8 times)
\item
  \textbf{André Calero Valdez}, AK Schaar, J Bender, S Aghassi, G Schuh,
  M Ziefle (2016). Social media applications for knowledge exchange in
  organizations. In: \emph{Innovations in Knowledge Management}, 147-176
  (cited 8 times)
\item
  \textbf{André Calero Valdez}, M Dehmer, A Holzinger (2016).
  Application of graph entropy for knowledge discovery and data mining
  in bibliometric data. In: \emph{Mathematical Foundations and
  Applications of Graph Entropy}, 6, 174 (cited 7 times)
\item
  T Hamann, AK Schaar, \textbf{André Calero Valdez}, M Ziefle (2016).
  Strategic knowledge management for interdisciplinary teams-overcoming
  barriers of interdisciplinary work via an online portal approach. In:
  \emph{International Conference on Human Interface and the Management
  of Information} (cited 7 times)
\item
  \textbf{André Calero Valdez}, P Brauner, M Ziefle, TW Kuhlen, M
  Sedlmair (2016). Human factors in information visualization and
  decision support systems. In: \emph{Mensch und Computer},
  2016--Workshopband (cited 6 times)
\item
  P Brauner, \textbf{André Calero Valdez}, R Philipsen, M Ziefle (2016).
  On studying human factors in complex cyber-physical systems. In:
  \emph{Mensch und Computer}, 2016--Workshopband (cited 5 times)
\item
  MA Yazdi, \textbf{André Calero Valdez}, L Lichtschlag, M Ziefle, J
  Borchers (2016). Visualizing opportunities of collaboration in large
  research organizations. In: \emph{International Conference on HCI in
  Business, Government, and Organizations} (cited 3 times)
\item
  G Schuh, A Bräkling, \textbf{André Calero Valdez}, AK Schaar, M Ziefle
  (2016). Using Liferay as an Interdisciplinary Scientific Collaboration
  Portal. In: \emph{Social Computing and Social Media:}, 8th
  International Conference, SCSM 2016~ (cited 3 times)
\item
  \textbf{André Calero Valdez}, P Brauner, M Ziefle (2016). Preparing
  Production Systems for the Internet of Things The Potential of
  Socio-Technical Approaches in Dealing with Complexity. In:
  \emph{Competitive Manufacturing (COMA)} (cited 2 times)
\item
  T Thiele, S Schröder, \textbf{André Calero Valdez}, C Jooß, A Richert,
  M Ziefle, et al. (2016). Unterstützung interdisziplinärer integration
  am Beispiel einer Exzellenzcluster-Community. In: \emph{Automation,
  Communication and Cybernetics in Science and Engineering}, 2015~
  (cited 1 times)
\item
  S Abels, C Greven, \textbf{André Calero Valdez}, U Schroeder, M Ziefle
  (2016). Graph Complexity in visual recommender systems for scientific
  literature. In: \emph{Mensch und Computer}, 2016--Workshopband (cited
  1 times)
\item
  L Platte, S Freitag, L Schubert, J Molis, \textbf{André Calero
  Valdez}, M Ziefle (2016). Trustworthiness of visualizations of
  mobility-induced CO2 emissions. In: \emph{Mensch und Computer},
  2016--Workshopband
\item
  \textbf{André Calero Valdez}, P Brauner, AK Schaar, A Holzinger, M
  Ziefle (2015). Reducing complexity with simplicity-usability methods
  for industry 4.0. In: \emph{Proceedings}, 19th Triennial Congress of
  the IEA 9, 14 (cited 42 times)
\item
  S Bruns, \textbf{André Calero Valdez}, C Greven, M Ziefle, U Schroeder
  (2015). What should i read next? a personalized visual publication
  recommender system. In: \emph{International Conference on Human
  Interface and the Management of Information} (cited 14 times)
\item
  \textbf{André Calero Valdez}, S Bruns, C Greven, U Schroeder, M Ziefle
  (2015). What do my colleagues know? dealing with cognitive complexity
  in organizations through visualizations. In: \emph{International
  Conference on Learning and Collaboration Technologies}, 449-459 (cited
  8 times)
\item
  \textbf{André Calero Valdez}, D Özdemir, MA Yazdi, AK Schaar, M Ziefle
  (2015). Orchestrating collaboration-using visual collaboration
  suggestion for steering of research clusters. In: \emph{Procedia
  Manufacturing}, 3, 363-370 (cited 7 times)
\item
  AK Schaar, \textbf{André Calero Valdez}, M Ziefle (2015).
  User-centered design of business communities. The influence of
  diversity factors on motives to use communities in professional
  settings. In: \emph{Procedia Manufacturing}, 3, 645-652 (cited 4
  times)
\item
  \textbf{André Calero Valdez}, M Ziefle (2015). Older Users' Rejection
  of Mobile Health Apps a Case for a Stand-Alone Device?. In:
  \emph{International Conference on Human Aspects of IT for the Aged
  Population}, 38-49 (cited 3 times)
\item
  AK Schaar, \textbf{André Calero Valdez}, M Ziefle (2015).
  Nutzungsmotivation von sozialen Netzwerken im Arbeitskontext. In:
  \emph{Exploring Demographics}, 657-666 (cited 2 times)
\item
  J Bender, \textbf{André Calero Valdez}, J Hofmann, M Kerger, AK Oprée,
  AK Schaar, et al. (2015). Das Projekt iNec. In: \emph{Zukunft
  gestalten: Soziale Technologien in Organisationen in Zeiten des
  demographischen Wandels}
\item
  AK Löcker, D Eraßme, EM Jakobs, AK Schaar, \textbf{André Calero
  Valdez}, M Ziefle (2014). Yet another platform? Motivational factors
  for using online communities in business contexts. In: \emph{Advances
  in the Ergonomics in Manufacturing: Managing the enterprise of the
  Future} (cited 16 times)
\item
  AK Schaar, \textbf{André Calero Valdez}, M Ziefle, D Eraßme, AK
  Löcker, EM Jakobs (2014). Reasons for using social networks
  professionally. In: \emph{International Conference on Social Computing
  and Social Media}, 385-396 (cited 13 times)
\item
  \textbf{André Calero Valdez}, AK Schaar, M Ziefle, A Holzinger (2014).
  Enhancing interdisciplinary cooperation by social platforms. In:
  \emph{International Conference on Human Interface and the Management
  of Information} (cited 12 times)
\item
  C Jooß, F Welter, I Leisten, A Richert, AK Schaar, \textbf{André
  Calero Valdez}, EM Nick, et al. (2014). Scientific cooperation
  engineering in the cluster of excellence integrative production
  technology for high-wage countries at RWTH Aachen University. In:
  \emph{Automation, Communication and Cybernetics in Science and
  Engineering}, 2013~ (cited 12 times)
\item
  A Holzinger, M Schwarz, B Ofner, F Jeanquartier, \textbf{André Calero
  Valdez}, et al. (2014). Towards interactive visualization of
  longitudinal data to support knowledge discovery on multi-touch tablet
  computers. In: \emph{International Conference on Availability,
  Reliability, and Security}, 124-137 (cited 3 times)
\item
  A Holzinger, B Sommerauer, P Spitzer, S Juric, B Zalik, M Debevc, et
  al. (2014). Mobile computing is not always advantageous: lessons
  learned from a real-world case study in a hospital. In:
  \emph{International Conference on Availability, Reliability, and
  Security}, 110-123 (cited 3 times)
\item
  \textbf{André Calero Valdez}, AK Schaar, T Vaegs, T Thiele, T
  Kowalski, S Aghassi, et al. (2014). Scientific Cooperation Engineering
  Making Interdisciplinary Knowledge Available within Research
  Facilities and to External Stakeholders. In: \emph{Proceedings of
  the}, 10th International Conference on Webometrics~ (cited 3 times)
\item
  \textbf{André Calero Valdez} (2014). Technology acceptance and
  diabetes : User centered design of small screen devices for diabetes
  patients. In: \emph{Apprimus Verlag, Aachen} (cited 1 times)
\item
  P Brauner, \textbf{André Calero Valdez}, U Schroeder, M Ziefle (2013).
  Increase physical fitness and create health awareness through
  exergames and gamification. In: \emph{International Conference on
  Human Factors in Computing and Informatics}, 349-362 (cited 65 times)
\item
  A Holzinger, B Ofner, C Stocker, \textbf{André Calero Valdez}, AK
  Schaar, M Ziefle, et al. (2013). On graph entropy measures for
  knowledge discovery from publication network data. In:
  \emph{International Conference on Availability, Reliability, and
  Security}, 354-362 (cited 39 times)
\item
  AK Schaar, \textbf{André Calero Valdez}, M Ziefle (2013). The impact
  of user diversity on the willingness to disclose personal information
  in social network services. In: \emph{International Conference on
  Human Factors in Computing and Informatics}, 174-193 (cited 27 times)
\item
  \textbf{André Calero Valdez}, AK Schaar, M Ziefle (2013). Personality
  influences on etiquette requirements for social media in the work
  context. In: \emph{International Conference on Human Factors in
  Computing and Informatics}, 427-446 (cited 17 times)
\item
  G Schuh, S Aghassi, \textbf{André Calero Valdez} (2013). Supporting
  technology transfer via web-based platforms. , 2013 Proceedings of
  PICMET'13: Technology Management in the IT-Driven~ (cited 13 times)
\item
  AK Schaar, \textbf{André Calero Valdez}, M Ziefle (2013). Publication
  network visualization as an approach for interdisciplinary innovation
  management. In: \emph{IEEE International Professonal Communication},
  2013 Conference, 1-8 (cited 10 times)
\item
  AP Schmitz-Urban, J Bender, G Gudergan, AK Schaar, \textbf{André
  Calero Valdez}, et al. (2013). Einsatz von Experten-Communitys zur
  Sicherung der Innovationsfähigkeit im demografischen Wandel. In:
  \emph{Innovationsfähigkeit im demografischen Wandel : Beiträge der
  Forschungspartner} (cited 7 times)
\item
  \textbf{André Calero Valdez}, AK Schaar, M Ziefle (2013). Measuring
  Interdisciplinary Performance By Analyzing Mixed Node Publication
  Networks. In: \emph{Translational twists and turns: Science as a
  socio-economic endeavor}
\item
  AK Schaar, \textbf{André Calero Valdez}, D Eraßme, AK Löcker, J
  Hofmann, et al. (2013). Wo ist all das Wissen hin? : Förderung der
  Kommunikation zwischen den Generationen in Unternehmen zur
  Wissenssicherung. In: \emph{Zeitschrift für innovative
  Arbeitsgestaltung und Prävention}, 3 (1), 22-23
\item
  \textbf{André Calero Valdez}, AK Schaar, M Ziefle, A Holzinger, S
  Jeschke, C Brecher (2012). Using mixed node publication network graphs
  for analyzing success in interdisciplinary teams. In:
  \emph{International Conference on Active Media Technology}, 606-617
  (cited 39 times)
\item
  \textbf{André Calero Valdez}, AK Schaar, M Ziefle (2012). State of the
  (net) work address Developing criteria for applying social networking
  to the work environment. In: \emph{Work}, 41 (Supplement 1), 3459-3467
  (cited 20 times)
\item
  AK Schaar, \textbf{André Calero Valdez}, M Ziefle (2012). Social media
  for the ehealth context. A requirement assessment. In: \emph{Advances
  in Human Aspects of Healthcare}, 79 (cited 12 times)
\item
  F Alagöz, M Ziefle, W Wilkowska, \textbf{André Calero Valdez} (2011).
  Openness to accept medical technology-a cultural view. In:
  \emph{Symposium of the Austrian HCI and Usability Engineering Group},
  151-170 (cited 37 times)
\item
  A Holzinger, S Dorner, M Födinger, \textbf{André Calero Valdez}, M
  Ziefle (2010). Chances of increasing youth health awareness through
  mobile wellness applications. In: \emph{Symposium of the Austrian HCI
  and Usability Engineering Group}, 71-81 (cited 74 times)
\item
  F Alagöz, \textbf{André Calero Valdez}, W Wilkowska, M Ziefle, S
  Dorner, A Holzinger (2010). From cloud computing to mobile Internet,
  from user focus to culture and hedonism: the crucible of mobile health
  care and wellness applications. , 5th International Conference on
  Pervasive Computing and Applications, 38-45 (cited 63 times)
\item
  \textbf{André Calero Valdez}, M Ziefle, F Alagöz, A Holzinger (2010).
  Mental models of menu structures in diabetes assistants. In:
  \emph{Computers Helping People with Special Needs}, 584-591 (cited 25
  times)
\item
  \textbf{André Calero Valdez}, M Ziefle, A Horstmann, D Herding, U
  Schroeder (2010). Task performance in mobile and ambient interfaces.
  Does size matter for usability of electronic diabetes assistants?. ,
  2010 International Conference on Information Society, 514-521 (cited 9
  times)
\item
  A Holzinger, S Dorner, M Födinger, M Ziefle, \textbf{André Calero
  Valdez} (2010). Motivational features of a mobile web application
  promoting a healthy lifestyle.. , 24th BCS Conference on Human
  Computer Interaction (cited 1 times)
\item
  \textbf{André Calero Valdez}, M Ziefle, A Horstmann, D Herding, U
  Schröder (2010). Mobile devices used for medical applications:
  Insights won from a usability study with diabetes patients. In:
  \emph{International Journal for Digital Society (IJDS)}, 1 (4),
  298-307
\item
  \textbf{André Calero Valdez}, M Ziefle, A Horstmann, D Herding, U
  Schroeder (2009). Effects of aging and domain knowledge on usability
  in small screen devices for diabetes patients. In: \emph{HCI and
  Usability for e-Inclusion}, 366-386 (cited 23 times)
\end{enumerate}


\end{document}